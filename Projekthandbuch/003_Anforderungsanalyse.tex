\section{Anforderungsanalyse}

\subsection{Funktionale Anforderungen}

Die Funktionalen Anforderungen beantworten die Frage Was das Programm leisten soll.

\begin{itemize}
  \item Erfassung von Meldungen über Schädlingsbefall
  \item Speichern der Meldungen
  \item Anzeige und Sortierung dieser Meldungen
    \item Weiterleitung der Meldungen an die Grünflächenabteilung
    \item Benachrichtigungen an Nutzer über Bearbeitungsstatus
\end{itemize}

\subsection{Nicht-funktionale Anforderungen}

Die Nichtfunktionalen Anforderungen beantworten die Frage: Wie soll das Programm diese Anforderungen erfüllen?

\begin{itemize}
    \item Benutzerfreundlichkeit
    \item Performance und Sicherheit
  \item  Einfache Bedienbarkeit der App
  \item Leichte Wartbarkeit der App
  \item Leichte Erweiterbarkeit der App
  \item Zuverlässiges Funktionieren der App
  \item Datensicherheit und Datenschutz
  \item Schnelle Reaktionszeiten
  \item Niedriger Datenverbrauch im Mobilbetrieb
\end{itemize}

\subsection{Technische Anforderungen}
\begin{itemize}
    \item Android-Kompatibilität.
    \item Daten On-Premises, keine Cloud
\end{itemize}

\newpage

\subsection{User Stories}
\label{subsec:userStories}
\begin{table}[ht]
    \centering
    \begin{tabularx}{\textwidth}{|X|X|X|} 
        \hline
        Als [Rolle] & möchte ich [Funktionalität] & damit [Grund]\\
        \hline
        Mieter / Eigentümer & Schädlingsbefall / Unkrautbewuchs melden können & die Gärtner diesen zeitnah beseitigen\\
        \hline
        Landschaftsarchitekt & Mietermeldungen einsehen können & zu analysieren und Handlungsanweisungen geben zu können\\
        \hline
        Landschaftsarchitekt & Meldungen als Arbeitsanweisung an mein Team weiterleiten können & arbeiten zu koordinieren\\
        \hline
        Landschaftsarchitekt & einsehen können, woran mein Team gerade arbeitet & zu priorisieren und gegebenenfalls einzugreifen\\
        \hline
        Gärtner & einsehen können, wo ich arbeiten soll & den Einsatzort schnell finde \\
        \hline
        Gärtner & einsehen können, woran ich arbeiten soll & die Arbeit vorbereiten kann (passenede Herbi/Pestizide) \\
        \hline
        Abteilungsleiter Aussenarbeiten& einsehen können, woran die verschiedenen Teams gerade arbeiten & Arbeiten zu koordinieren und Absprachen mit den Landschaftsarchitekten halten zu können.\\
        \hline
        Abteilungsleiter Aussenarbeiten& die Meldungen über Befall und Bewuchs einsehen können & eventuelle Muster zu erkennen und großflächige Ausbreitung von Unkraut und Schädlingen zu verhindern.\\
        \hline
    \end{tabularx}
    \caption{Ausgewählte User Stories}
    \label{tab:userStories}
\end{table}