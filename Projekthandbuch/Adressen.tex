address-form.fxml, die zusammen mit den anderen Komponenten für die Formularansicht verwendet wird:

<?xml version="1.0" encoding="UTF-8"?>
<?import javafx.scene.control.TextField?>
<?import javafx.scene.layout.VBox?>

<?import javafx.scene.control.Label?>
<VBox xmlns:fx="http://javafx.com/fxml/1" fx:controller="com.mch.philetairusadmintool.AddressFormController" spacing="10">

    <Label text="Adresse" styleClass="section-header"/>
    <TextField fx:id="streetField" promptText="Straße"/>
    <TextField fx:id="zipField" promptText="PLZ"/>
    <TextField fx:id="cityField" promptText="Stadt"/>
</VBox>


der dazugehörige Controller, AddressFormController:

package com.mch.philetairusadmintool;

import javafx.fxml.FXML;
import javafx.scene.control.TextField;

public class AddressFormController {

    @FXML
    private TextField streetField;

    @FXML
    private TextField zipField;

    @FXML
    private TextField cityField;

    public String getStreet() {
        return streetField.getText();
    }

    public String getZip() {
        return zipField.getText();
    }

    public String getCity() {
        return cityField.getText();
    }
}

und die Klasse, die später diese Daten erhalten soll (eingeladen aus dem shared Projekt in jitpack):

plugins {
  id 'java'
  id 'application'
  id 'org.javamodularity.moduleplugin' version '1.8.12'
  id 'org.openjfx.javafxplugin' version '0.0.13'
  id 'org.beryx.jlink' version '2.25.0'
  id 'jacoco'
}

group 'com.mch'
version '1.0-SNAPSHOT'

repositories {
  mavenCentral()
  maven { url "https://jitpack.io" }
}

ext {
  junitVersion = '5.10.2'
}

sourceCompatibility = '17'
targetCompatibility = '17'

tasks.withType(JavaCompile) {
  options.encoding = 'UTF-8'
  options.forkOptions.executable = 'javac'
}
application {
  mainModule = 'com.mch.philetairusadmintool'
  mainClass = 'com.mch.philetairusadmintool.MainApp'
}

javafx {
  version = '17.0.6'
  modules = ['javafx.controls', 'javafx.fxml']
}

dependencies {
  implementation('org.controlsfx:controlsfx:11.2.1')
  implementation('com.dlsc.formsfx:formsfx-core:11.6.0') {
    exclude(group: 'org.openjfx')
  }
  implementation('net.synedra:validatorfx:0.5.0') {
    exclude(group: 'org.openjfx')
  }
  implementation('org.kordamp.ikonli:ikonli-javafx:12.3.1')
  implementation('org.kordamp.bootstrapfx:bootstrapfx-core:0.4.0')

  implementation('com.github.MCHDevelopment:philetiptipshared:main-SNAPSHOT')

  testImplementation("org.junit.jupiter:junit-jupiter-api:${junitVersion}")
  testRuntimeOnly("org.junit.jupiter:junit-jupiter-engine:${junitVersion}")
}

test {
useJUnitPlatform()}

jlink {
  imageZip = project.file("${buildDir}/distributions/app-${javafx.platform.classifier}.zip")
  options = ['--strip-debug', '--compress', '2', '--no-header-files', '--no-man-pages']
  launcher {
    name = 'app'
  }
}

jlinkZip {
  group = 'distribution'
}

jacocoTestReport {
  dependsOn test // Tests müssen abgeschlossen sein, bevor der Report erstellt wird
}

repositories {
  maven {
    url "https://jitpack.io"
    credentials {
      username = authToken
      password = "" // kein Passwort nötig
    }
  }
}

Und die Struktur der Datenbank, in der diese Daten später erfasst werden sollen:

CREATE TABLE adressen (
    adressId INT AUTO_INCREMENT PRIMARY KEY,
    strasse VARCHAR(100),
    hausnummer VARCHAR(10),
    plz VARCHAR(10),
    ort VARCHAR(100)
);